\documentclass[]{article}
\usepackage{lmodern}
\usepackage{amssymb,amsmath}
\usepackage{ifxetex,ifluatex}
\usepackage{fixltx2e} % provides \textsubscript
\ifnum 0\ifxetex 1\fi\ifluatex 1\fi=0 % if pdftex
  \usepackage[T1]{fontenc}
  \usepackage[utf8]{inputenc}
\else % if luatex or xelatex
  \ifxetex
    \usepackage{mathspec}
  \else
    \usepackage{fontspec}
  \fi
  \defaultfontfeatures{Ligatures=TeX,Scale=MatchLowercase}
\fi
% use upquote if available, for straight quotes in verbatim environments
\IfFileExists{upquote.sty}{\usepackage{upquote}}{}
% use microtype if available
\IfFileExists{microtype.sty}{%
\usepackage{microtype}
\UseMicrotypeSet[protrusion]{basicmath} % disable protrusion for tt fonts
}{}
\usepackage[margin=1in]{geometry}
\usepackage{hyperref}
\hypersetup{unicode=true,
            pdftitle={Practica4.r},
            pdfauthor={Equipo},
            pdfborder={0 0 0},
            breaklinks=true}
\urlstyle{same}  % don't use monospace font for urls
\usepackage{color}
\usepackage{fancyvrb}
\newcommand{\VerbBar}{|}
\newcommand{\VERB}{\Verb[commandchars=\\\{\}]}
\DefineVerbatimEnvironment{Highlighting}{Verbatim}{commandchars=\\\{\}}
% Add ',fontsize=\small' for more characters per line
\usepackage{framed}
\definecolor{shadecolor}{RGB}{248,248,248}
\newenvironment{Shaded}{\begin{snugshade}}{\end{snugshade}}
\newcommand{\AlertTok}[1]{\textcolor[rgb]{0.94,0.16,0.16}{#1}}
\newcommand{\AnnotationTok}[1]{\textcolor[rgb]{0.56,0.35,0.01}{\textbf{\textit{#1}}}}
\newcommand{\AttributeTok}[1]{\textcolor[rgb]{0.77,0.63,0.00}{#1}}
\newcommand{\BaseNTok}[1]{\textcolor[rgb]{0.00,0.00,0.81}{#1}}
\newcommand{\BuiltInTok}[1]{#1}
\newcommand{\CharTok}[1]{\textcolor[rgb]{0.31,0.60,0.02}{#1}}
\newcommand{\CommentTok}[1]{\textcolor[rgb]{0.56,0.35,0.01}{\textit{#1}}}
\newcommand{\CommentVarTok}[1]{\textcolor[rgb]{0.56,0.35,0.01}{\textbf{\textit{#1}}}}
\newcommand{\ConstantTok}[1]{\textcolor[rgb]{0.00,0.00,0.00}{#1}}
\newcommand{\ControlFlowTok}[1]{\textcolor[rgb]{0.13,0.29,0.53}{\textbf{#1}}}
\newcommand{\DataTypeTok}[1]{\textcolor[rgb]{0.13,0.29,0.53}{#1}}
\newcommand{\DecValTok}[1]{\textcolor[rgb]{0.00,0.00,0.81}{#1}}
\newcommand{\DocumentationTok}[1]{\textcolor[rgb]{0.56,0.35,0.01}{\textbf{\textit{#1}}}}
\newcommand{\ErrorTok}[1]{\textcolor[rgb]{0.64,0.00,0.00}{\textbf{#1}}}
\newcommand{\ExtensionTok}[1]{#1}
\newcommand{\FloatTok}[1]{\textcolor[rgb]{0.00,0.00,0.81}{#1}}
\newcommand{\FunctionTok}[1]{\textcolor[rgb]{0.00,0.00,0.00}{#1}}
\newcommand{\ImportTok}[1]{#1}
\newcommand{\InformationTok}[1]{\textcolor[rgb]{0.56,0.35,0.01}{\textbf{\textit{#1}}}}
\newcommand{\KeywordTok}[1]{\textcolor[rgb]{0.13,0.29,0.53}{\textbf{#1}}}
\newcommand{\NormalTok}[1]{#1}
\newcommand{\OperatorTok}[1]{\textcolor[rgb]{0.81,0.36,0.00}{\textbf{#1}}}
\newcommand{\OtherTok}[1]{\textcolor[rgb]{0.56,0.35,0.01}{#1}}
\newcommand{\PreprocessorTok}[1]{\textcolor[rgb]{0.56,0.35,0.01}{\textit{#1}}}
\newcommand{\RegionMarkerTok}[1]{#1}
\newcommand{\SpecialCharTok}[1]{\textcolor[rgb]{0.00,0.00,0.00}{#1}}
\newcommand{\SpecialStringTok}[1]{\textcolor[rgb]{0.31,0.60,0.02}{#1}}
\newcommand{\StringTok}[1]{\textcolor[rgb]{0.31,0.60,0.02}{#1}}
\newcommand{\VariableTok}[1]{\textcolor[rgb]{0.00,0.00,0.00}{#1}}
\newcommand{\VerbatimStringTok}[1]{\textcolor[rgb]{0.31,0.60,0.02}{#1}}
\newcommand{\WarningTok}[1]{\textcolor[rgb]{0.56,0.35,0.01}{\textbf{\textit{#1}}}}
\usepackage{graphicx,grffile}
\makeatletter
\def\maxwidth{\ifdim\Gin@nat@width>\linewidth\linewidth\else\Gin@nat@width\fi}
\def\maxheight{\ifdim\Gin@nat@height>\textheight\textheight\else\Gin@nat@height\fi}
\makeatother
% Scale images if necessary, so that they will not overflow the page
% margins by default, and it is still possible to overwrite the defaults
% using explicit options in \includegraphics[width, height, ...]{}
\setkeys{Gin}{width=\maxwidth,height=\maxheight,keepaspectratio}
\IfFileExists{parskip.sty}{%
\usepackage{parskip}
}{% else
\setlength{\parindent}{0pt}
\setlength{\parskip}{6pt plus 2pt minus 1pt}
}
\setlength{\emergencystretch}{3em}  % prevent overfull lines
\providecommand{\tightlist}{%
  \setlength{\itemsep}{0pt}\setlength{\parskip}{0pt}}
\setcounter{secnumdepth}{0}
% Redefines (sub)paragraphs to behave more like sections
\ifx\paragraph\undefined\else
\let\oldparagraph\paragraph
\renewcommand{\paragraph}[1]{\oldparagraph{#1}\mbox{}}
\fi
\ifx\subparagraph\undefined\else
\let\oldsubparagraph\subparagraph
\renewcommand{\subparagraph}[1]{\oldsubparagraph{#1}\mbox{}}
\fi

%%% Use protect on footnotes to avoid problems with footnotes in titles
\let\rmarkdownfootnote\footnote%
\def\footnote{\protect\rmarkdownfootnote}

%%% Change title format to be more compact
\usepackage{titling}

% Create subtitle command for use in maketitle
\providecommand{\subtitle}[1]{
  \posttitle{
    \begin{center}\large#1\end{center}
    }
}

\setlength{\droptitle}{-2em}

  \title{Practica4.r}
    \pretitle{\vspace{\droptitle}\centering\huge}
  \posttitle{\par}
    \author{Equipo}
    \preauthor{\centering\large\emph}
  \postauthor{\par}
      \predate{\centering\large\emph}
  \postdate{\par}
    \date{2019-10-04}


\begin{document}
\maketitle

\begin{Shaded}
\begin{Highlighting}[]
\KeywordTok{library}\NormalTok{(readr)}
\KeywordTok{library}\NormalTok{(dplyr)}
\end{Highlighting}
\end{Shaded}

\begin{verbatim}
## 
## Attaching package: 'dplyr'
\end{verbatim}

\begin{verbatim}
## The following objects are masked from 'package:stats':
## 
##     filter, lag
\end{verbatim}

\begin{verbatim}
## The following objects are masked from 'package:base':
## 
##     intersect, setdiff, setequal, union
\end{verbatim}

\begin{Shaded}
\begin{Highlighting}[]
\KeywordTok{library}\NormalTok{(tidyverse)}
\end{Highlighting}
\end{Shaded}

\begin{verbatim}
## -- Attaching packages ------------------------------------------------ tidyverse 1.2.1 --
\end{verbatim}

\begin{verbatim}
## v ggplot2 3.2.1     v purrr   0.3.2
## v tibble  2.1.3     v stringr 1.4.0
## v tidyr   1.0.0     v forcats 0.4.0
\end{verbatim}

\begin{verbatim}
## -- Conflicts --------------------------------------------------- tidyverse_conflicts() --
## x dplyr::filter() masks stats::filter()
## x dplyr::lag()    masks stats::lag()
\end{verbatim}

\begin{Shaded}
\begin{Highlighting}[]
\KeywordTok{library}\NormalTok{(ggplot2)}
\CommentTok{#1. Crear un nuevo proyecto denominado practica 4.}

\CommentTok{# 2. Mediante la libreria readr, o mediante los menus de RStudio, leer los datasets sleep.csv y activities.csv}
\CommentTok{# ambos archivos deben estar previamente en la carpeta del proyecto creado}
\NormalTok{activities <-}\StringTok{ }\KeywordTok{read_csv}\NormalTok{(}\StringTok{"C:/Users/Equipo/Desktop/CUNEF/Programacion con R/Proyecto/Practica 4/Practica 4/activities.csv"}\NormalTok{)}
\end{Highlighting}
\end{Shaded}

\begin{verbatim}
## Parsed with column specification:
## cols(
##   de = col_datetime(format = ""),
##   a = col_datetime(format = ""),
##   `from (manual)` = col_logical(),
##   `to (manual)` = col_logical(),
##   Timezone = col_character(),
##   `Activity type` = col_character(),
##   Data = col_character(),
##   GPS = col_character(),
##   Modified = col_datetime(format = "")
## )
\end{verbatim}

\begin{Shaded}
\begin{Highlighting}[]
\NormalTok{sleep <-}\StringTok{ }\KeywordTok{read_csv}\NormalTok{(}\StringTok{"C:/Users/Equipo/Desktop/CUNEF/Programacion con R/Proyecto/Practica 4/Practica 4/sleep.csv"}\NormalTok{)}
\end{Highlighting}
\end{Shaded}

\begin{verbatim}
## Parsed with column specification:
## cols(
##   de = col_datetime(format = ""),
##   a = col_datetime(format = ""),
##   `ligero (s)` = col_double(),
##   `profundo (s)` = col_double(),
##   `Rem (seg)` = col_double(),
##   `despierto (s)` = col_double(),
##   despertar = col_double(),
##   `Duration to sleep (s)` = col_double(),
##   `Duration to wake up (s)` = col_double(),
##   `Snoring (s)` = col_double(),
##   `Snoring episodes` = col_double(),
##   `Average heart rate` = col_double(),
##   `Heart rate (min)` = col_double(),
##   `Heart rate (max)` = col_double()
## )
\end{verbatim}

\begin{Shaded}
\begin{Highlighting}[]
\CommentTok{# 3.Comprobar el contenido con View y contar cuantos NAs hay en la columna GPS del dataset activities}
\KeywordTok{View}\NormalTok{(activities)}
\NormalTok{NAs<-}\KeywordTok{sum}\NormalTok{(}\KeywordTok{is.na}\NormalTok{(activities}\OperatorTok{$}\NormalTok{GPS))}


\CommentTok{# 4. Crear un objeto R denominado act_new que contenga solo las variables}
\CommentTok{# siguientes: 1,2,5-6}
\NormalTok{act_new <-}\StringTok{ }\KeywordTok{select}\NormalTok{(activities, de, a, Timezone, }\StringTok{`}\DataTypeTok{Activity type}\StringTok{`}\NormalTok{)}
\CommentTok{# 5. Renombrar la variable 'Activity type' con el nombre 'tipo' y la variable 'Time zone' como 'ciudad'}
\NormalTok{act_new <-}\StringTok{ }\KeywordTok{rename}\NormalTok{(act_new, }\DataTypeTok{ciudad =}\NormalTok{ Timezone, }\DataTypeTok{tipo =} \StringTok{`}\DataTypeTok{Activity type}\StringTok{`}\NormalTok{)}

\CommentTok{# 6. Realizar un recuento de tipo de actividad con summary. Para ello}
\CommentTok{# debes transformar previamente la variable tipo a factor con as.factor.}
\CommentTok{# Crea un grafico de barras con dicha variable par visualizar las frecuencias.}
\CommentTok{# Haz lo mismo para la variable ciudad}
\NormalTok{act_new}\OperatorTok{$}\NormalTok{tipo <-}\StringTok{ }\KeywordTok{as.factor}\NormalTok{(act_new}\OperatorTok{$}\NormalTok{tipo)}
\KeywordTok{str}\NormalTok{(act_new)}
\end{Highlighting}
\end{Shaded}

\begin{verbatim}
## Classes 'spec_tbl_df', 'tbl_df', 'tbl' and 'data.frame': 312 obs. of  4 variables:
##  $ de    : POSIXct, format: "2018-11-05 08:30:00" "2018-11-05 10:02:00" ...
##  $ a     : POSIXct, format: "2018-11-05 09:15:00" "2018-11-05 10:21:00" ...
##  $ ciudad: chr  "Europe/Madrid" "Europe/Madrid" "Europe/Madrid" "Europe/Madrid" ...
##  $ tipo  : Factor w/ 10 levels "Cycling","Dancing",..: 8 10 10 10 10 10 10 10 10 10 ...
##  - attr(*, "spec")=
##   .. cols(
##   ..   de = col_datetime(format = ""),
##   ..   a = col_datetime(format = ""),
##   ..   `from (manual)` = col_logical(),
##   ..   `to (manual)` = col_logical(),
##   ..   Timezone = col_character(),
##   ..   `Activity type` = col_character(),
##   ..   Data = col_character(),
##   ..   GPS = col_character(),
##   ..   Modified = col_datetime(format = "")
##   .. )
\end{verbatim}

\begin{Shaded}
\begin{Highlighting}[]
\KeywordTok{summary}\NormalTok{(act_new}\OperatorTok{$}\NormalTok{tipo)}
\end{Highlighting}
\end{Shaded}

\begin{verbatim}
##  Cycling  Dancing   Hiking    Other  Pilates  Running      Ski Swimming 
##        7        1        1       12        8       49        4        9 
##   Tennis  Walking 
##        4      217
\end{verbatim}

\begin{Shaded}
\begin{Highlighting}[]
\KeywordTok{ggplot}\NormalTok{(}\DataTypeTok{data =}\NormalTok{ act_new)}\OperatorTok{+}
\StringTok{  }\KeywordTok{geom_bar}\NormalTok{(}\DataTypeTok{mapping =} \KeywordTok{aes}\NormalTok{(}\DataTypeTok{x =}\NormalTok{ tipo), }\DataTypeTok{color =} \StringTok{"darkblue"}\NormalTok{)}
\end{Highlighting}
\end{Shaded}

\includegraphics{Practica4_files/figure-latex/unnamed-chunk-1-1.pdf}

\begin{Shaded}
\begin{Highlighting}[]
\CommentTok{#7. Filtrar los registros de act_new que correspondan con ciudad Amsterdam en otro objeto}
\CommentTok{# y lo mismo con Madrid. Con esos nuevos objetos determina los deportes que}
\CommentTok{# no se practican en Amsterdam y sí en Madrid y viceversa. Genera graficos para visualizar los resultados}

\NormalTok{Amsterdam <-}\StringTok{ }\KeywordTok{filter}\NormalTok{(act_new, ciudad }\OperatorTok{==}\StringTok{ "Europe/Amsterdam"}\NormalTok{)}

\NormalTok{Madrid <-}\StringTok{ }\KeywordTok{filter}\NormalTok{(act_new, ciudad }\OperatorTok{==}\StringTok{ "Europe/Madrid"}\NormalTok{)}

\KeywordTok{ggplot}\NormalTok{(}\DataTypeTok{data =}\NormalTok{ act_new)}\OperatorTok{+}
\StringTok{  }\KeywordTok{geom_bar}\NormalTok{(}\DataTypeTok{mapping =} \KeywordTok{aes}\NormalTok{(}\DataTypeTok{x =}\NormalTok{ ciudad), }\DataTypeTok{color =} \StringTok{"darkblue"}\NormalTok{)}
\end{Highlighting}
\end{Shaded}

\includegraphics{Practica4_files/figure-latex/unnamed-chunk-1-2.pdf}

\begin{Shaded}
\begin{Highlighting}[]
\CommentTok{####Deportes que estan unicamente en Amsterdam}
\NormalTok{a <-}\StringTok{ }\KeywordTok{data.frame}\NormalTok{(}\KeywordTok{summary}\NormalTok{(Amsterdam}\OperatorTok{$}\NormalTok{tipo), }\KeywordTok{summary}\NormalTok{(Madrid}\OperatorTok{$}\NormalTok{tipo))}
\NormalTok{a <-}\StringTok{ }\KeywordTok{rename}\NormalTok{(a, }\DataTypeTok{Amsterdam =}\NormalTok{ summary.Amsterdam.tipo. , }\DataTypeTok{Madrid =}\NormalTok{ summary.Madrid.tipo.)}

\KeywordTok{summary}\NormalTok{(Amsterdam)}
\end{Highlighting}
\end{Shaded}

\begin{verbatim}
##        de                            a                      
##  Min.   :2019-03-19 07:15:00   Min.   :2019-03-19 08:15:00  
##  1st Qu.:2019-04-19 11:38:15   1st Qu.:2019-04-19 11:47:00  
##  Median :2019-07-17 19:36:30   Median :2019-07-17 19:54:00  
##  Mean   :2019-06-26 18:00:35   Mean   :2019-06-26 18:16:45  
##  3rd Qu.:2019-08-26 12:16:45   3rd Qu.:2019-08-26 12:36:45  
##  Max.   :2019-09-28 08:33:00   Max.   :2019-09-28 08:42:00  
##                                                             
##     ciudad               tipo   
##  Length:94          Walking:59  
##  Class :character   Running:25  
##  Mode  :character   Cycling: 6  
##                     Pilates: 3  
##                     Dancing: 1  
##                     Hiking : 0  
##                     (Other): 0
\end{verbatim}

\begin{Shaded}
\begin{Highlighting}[]
\NormalTok{Ams_dep <-}\StringTok{ }\KeywordTok{filter}\NormalTok{(act_new, ciudad }\OperatorTok{==}\StringTok{ "Europe/Amsterdam"}\NormalTok{, tipo }\OperatorTok{==}\StringTok{ "Dancing"}\NormalTok{)}


\KeywordTok{summary}\NormalTok{(Madrid)}
\end{Highlighting}
\end{Shaded}

\begin{verbatim}
##        de                            a                      
##  Min.   :2018-11-05 08:30:00   Min.   :2018-11-05 09:15:00  
##  1st Qu.:2018-11-28 15:14:00   1st Qu.:2018-11-28 15:34:00  
##  Median :2019-03-14 13:44:00   Median :2019-03-14 13:49:00  
##  Mean   :2019-03-10 21:18:05   Mean   :2019-03-10 21:36:48  
##  3rd Qu.:2019-04-26 07:26:00   3rd Qu.:2019-04-26 07:38:00  
##  Max.   :2019-09-23 13:37:00   Max.   :2019-09-23 13:49:00  
##                                                             
##     ciudad                tipo    
##  Length:213         Walking :157  
##  Class :character   Running : 24  
##  Mode  :character   Other   : 11  
##                     Swimming:  9  
##                     Pilates :  5  
##                     Tennis  :  4  
##                     (Other) :  3
\end{verbatim}

\begin{Shaded}
\begin{Highlighting}[]
\NormalTok{Mad_dep <-}\StringTok{ }\KeywordTok{filter}\NormalTok{(act_new, ciudad }\OperatorTok{==}\StringTok{ "Europe/Madrid"}\NormalTok{, tipo }\OperatorTok\StringTok{ }\KeywordTok{c}\NormalTok{(}\StringTok{"Hiking"}\NormalTok{, }\StringTok{"Other"}\NormalTok{, }\StringTok{"Ski"}\NormalTok{, }\StringTok{"Swimming"}\NormalTok{, }\StringTok{"Tennis"}\NormalTok{))}

\KeywordTok{ggplot}\NormalTok{(}\DataTypeTok{data =}\NormalTok{ Ams_dep)}\OperatorTok{+}
\StringTok{  }\KeywordTok{geom_bar}\NormalTok{(}\DataTypeTok{mapping =} \KeywordTok{aes}\NormalTok{(}\DataTypeTok{x =}\NormalTok{ tipo), }\DataTypeTok{color =} \StringTok{"darkblue"}\NormalTok{)}
\end{Highlighting}
\end{Shaded}

\includegraphics{Practica4_files/figure-latex/unnamed-chunk-1-3.pdf}

\begin{Shaded}
\begin{Highlighting}[]
\KeywordTok{ggplot}\NormalTok{(}\DataTypeTok{data =}\NormalTok{ Mad_dep)}\OperatorTok{+}
\StringTok{  }\KeywordTok{geom_bar}\NormalTok{(}\DataTypeTok{mapping =} \KeywordTok{aes}\NormalTok{(}\DataTypeTok{x =}\NormalTok{ tipo), }\DataTypeTok{color =} \StringTok{"pink"}\NormalTok{)}
\end{Highlighting}
\end{Shaded}

\includegraphics{Practica4_files/figure-latex/unnamed-chunk-1-4.pdf}

\begin{Shaded}
\begin{Highlighting}[]
\CommentTok{#8. Encontrar las fechas en las que se ha practicado bicicleta o pilates en Amsterdam en el año 2019}

\NormalTok{bike_pil <-}\StringTok{ }\KeywordTok{filter}\NormalTok{(Amsterdam, de }\OperatorTok{==}\StringTok{ }\NormalTok{tipo }\OperatorTok\StringTok{ }\KeywordTok{c}\NormalTok{(}\StringTok{"Cycling"}\NormalTok{, }\StringTok{"Pilates"}\NormalTok{))}
\KeywordTok{print}\NormalTok{(bike_pil}\OperatorTok{$}\NormalTok{de)}
\end{Highlighting}
\end{Shaded}

\begin{verbatim}
## POSIXct of length 0
\end{verbatim}

\begin{Shaded}
\begin{Highlighting}[]
\CommentTok{#9. Crear una nueva variable dif con los minutos de realización de cada actividad en Amsterdam}
\CommentTok{# y realizar una representación gráfica de los resultados con plot y determinar que deporte o deportes}
\CommentTok{# se han practicado durante dos horas o mas}

\NormalTok{minutes <-}\StringTok{ }\KeywordTok{data.frame}\NormalTok{(Amsterdam}\OperatorTok{$}\NormalTok{a }\OperatorTok{-}\StringTok{ }\NormalTok{Amsterdam}\OperatorTok{$}\NormalTok{de)}
\NormalTok{minutes <-}\StringTok{ }\KeywordTok{rename}\NormalTok{(minutes, }\DataTypeTok{minutes =}\NormalTok{ Amsterdam.a...Amsterdam.de)}
\KeywordTok{as.numeric}\NormalTok{(Amsterdam}\OperatorTok{$}\NormalTok{minutes)}
\end{Highlighting}
\end{Shaded}

\begin{verbatim}
## Warning: Unknown or uninitialised column: 'minutes'.
\end{verbatim}

\begin{verbatim}
## numeric(0)
\end{verbatim}

\begin{Shaded}
\begin{Highlighting}[]
\NormalTok{Amsterdam <-}\StringTok{ }\KeywordTok{cbind}\NormalTok{(Amsterdam, minutes)}


\CommentTok{#No he sabido hacer la agrupacion de datos por actividad de mas de 120 minutos pero si su}

\CommentTok{#10. Guardar el nuevo dataset en un archivo llamado "act_new.csv"}
\end{Highlighting}
\end{Shaded}


\end{document}
